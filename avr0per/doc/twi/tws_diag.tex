% todo: MCMD for recv data, repeated start
% todo: must check ERR before DACK
%\begin{tikzpicture}[text=black,node font=\small,>=stealth]%,>={scale=2}]
\begin{tikzpicture}[%
    text=black,
    node font=\small,
    >=stealth,
    arrow/.style={scale=2}]
  
  %
  \node (RDY) at (0,0) [rectangle, rounded corners=15pt, draw,
    minimum height=0.4in, minimum width=0.8in, thick]
        {\parbox[t]{0.6in}{\centering{READY}}};

  \node (D1) [left=5.0cm of RDY, diamond, draw] {};
  \draw[->] (RDY) -- node[above]{1:intr[ADR]} (D1);

  \node (S0) at ($(RDY.north)+(-1.0cm,0.5cm)$)
        [circle, draw, radius=8pt, fill] {};
  \draw[->] (S0) -- (RDY);

  % --- READ ------------------------

  \coordinate (RC) at ($(RDY) + (0,2.5cm)$);
  \coordinate (RL) at (D1 |- RC);

  \node (RD1) [above=0.9cm of D1, diamond, draw] {};
  \draw[->] (D1) -- node[right] {2:[R]} (RD1); % DIR set
  \coordinate (RD1a) at ($(RD1.east)+(0.5cm,0)$);
  \draw[->] (RD1) -- (RD1a) -- node[above,sloped,pos=0.3] {12:[else]/NACK} (RDY);

  \node (RD2) [diamond, draw] at (RC) {};
  \draw[->] (RD2) -- node[right,pos=0.3] {15:[STOP]} (RDY);

  \node (RE1) [left=0 of RD2, diamond, draw] {};

  % wait for data sent
  \node (WTS) [draw, rounded corners=15pt,
               minimum height=0.4in, minimum width=0.75in, thick]
        at ($0.6*(RL)+0.4*(RC)$) 
        {\parbox[c]{0.9in}{\centering{WAIT for DATA SENT}}};

  \draw[->,rounded corners]
        (RD1) -- (RL) -- node[above,pos=-0.4] {11:[OK]/ACK,put} (WTS);

  \draw[->] (WTS) -- node[above] {13:intr} (RE1);

  % wait to re-send
  \node (WST) [right=2.0cm of RD2, rounded corners=15pt,
    draw, minimum height=0.4in, minimum width=0.75in, thick]
        {\parbox[c]{0.8in}{\centering{WAIT FOR STATUS}}};

  \draw[->] (RD2) -- node[above] {14:[DIF]/put} (WST);

  \node (RD3) [right=2.0cm of WST, diamond, draw] {};
  \node (RD3a) [left=0cm of RD3, diamond, draw] {};

  \draw[->] (WST) -- node[above] {16:intr} (RD3a);

  \node (WFS) at (RD3 |- RDY) [rounded corners=15pt,
    draw, minimum height=0.4in, minimum width=0.75in, thick]
        {\parbox[c]{0.8in}{\centering{WAIT FOR STOP}}};

  \draw[->] (RD3) -- node[right,pos=0.3]{18:[NACK]} (WFS);
  \draw[->] (WFS) -- node[below,sloped] {4:intr[STOP]} (RDY);

  \coordinate (R2) at ($(RE1)+(0,-1.0cm)$);
  \draw[->] (RE1) -- node[left] {29:[BUSERR]} (R2) -- (WFS);

  \node (RD4) [above=0.7cm of RD3, diamond, draw] {};
  \draw[->] (RD3) -- node[right] {17:[ACK]} (RD4);

  \coordinate (R3) at ($(RD3a)+(0.0cm,-0.4cm)$);
  \draw[->,rounded corners]
   (RD3a) -- (R3) -- node[above,sloped] {19:[STOP]} (RDY);

  \coordinate (RX1) at ($(RD4) + (0,0.7cm)$);
  \draw[->,rounded corners] (RD4) -- (RX1)
            -- node[below] {20:[n$>$0]/put,n-{}-} (RX1 -| WTS) -- (WTS);

  % put zero out
  \draw[-] (RD4) -- node[below] {21:[n=0]/zero} (RD4 -| WTS);

  % --- WRITE -----------------------

  \coordinate (WC) at ($(RDY) + (0,-2.5cm)$);
  \coordinate (WL) at (D1 |- WC);

  \node (WD1) [below=0.9cm of D1, diamond, draw] {};
  \draw[->] (D1) -- node[right] {3:[W]} (WD1); % DIR clr
  \coordinate (WD1a) at ($(WD1.east)+(0.5cm,0)$);
  \draw[->,rounded corners]
   (WD1) -- (WD1a) -- node[below,sloped,pos=0.3] {32:[else]/NACK} (RDY);

  \node (WD2) [diamond, draw] at (WC) {};
  \draw[->] (WD2) -- node[right] {34:[STOP]} (RDY);

  % wait for data
  \node (WFI) [draw, rounded corners,
               minimum height=0.4in, minimum width=0.75in, thick]
        at ($0.6*(WL)+0.4*(WC)$) 
        {\parbox[c]{0.9in}{\centering{WAIT for DATA RCVD}}};

  \draw[->,rounded corners]
        (WD1) -- (WL) -- node[below,pos=-0.4] {31:[OK]/ACK} (WFI);

  \draw[->] (WFI) -- node[above] {33:intr} (WD2); % can be APIF or DIF

  \node (WD3) [right=0 of WD2, diamond, draw] {};

  \coordinate (W1) at ($(WD3) + (0,-1.0cm)$);
  \draw[->,rounded corners]
  (WD3) -- node[right] {35:[OK]/get,ACK} (W1) -- (WFI |- W1) -- (WFI);

  \draw[->,rounded corners]
  (WD3) -- node[above,pos=0.3] {36:[else]/NACK} (WD3 -| WFS) -- (WFS);

  \node (sig) [below=1cm of WL,rectangle,draw] {\tiny MattRW v230204a};
\end{tikzpicture}

\iffalse
Notes:
\begin{itemize}
BUSERR logic only works if the TWI Master is enabled.
\end{itemize}

\fi
