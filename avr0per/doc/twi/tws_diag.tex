% todo: MCMD for recv data, repeated start
% todo: must check ERR before DACK
%\begin{tikzpicture}[text=black,node font=\small,>=stealth]%,>={scale=2}]
\begin{tikzpicture}[%
    text=black,
    node font=\small,
    >=stealth,
    arrow/.style={scale=2}]
  
  %
  \node (RDY) at (0,0) [rectangle, rounded corners=15pt, draw,
    minimum height=0.4in, minimum width=0.8in, thick]
        {\parbox[t]{0.6in}{\centering{READY}}};

  \node (D1) [left=5.0cm of RDY, diamond, draw] {};

  % join for Wait for reStart
  \coordinate (J1) at ($(RDY.west)+(-4.0cm,0)$);
  \coordinate (J1t) at ($(J1)+(0,+0.1cm)$);
  \coordinate (J1b) at ($(J1)+(0,-0.3cm)$);
  \coordinate (J1x) at ($(J1)+(0,-0.2cm)$);
  \draw (J1t) edge [thick] (J1b);
  
  \draw[->] (RDY) -- node[above]{1:intr[ADR]} (D1 -| J1);
  \draw[->] (D1 -| J1) -- (D1);
  
  \node (S0) at ($(RDY.north)+(-1.0cm,0.5cm)$)
        [circle, draw, radius=8pt, fill] {};
  \draw[->] (S0) -- (RDY);

  % --- READ ------------------------

  \coordinate (RC) at ($(RDY) + (0,2.5cm)$);
  \coordinate (RL) at (D1 |- RC);

  \node (DS1) [above=0.9cm of D1, diamond, draw] {};
  \coordinate (DS1a) at ($(DS1.east)+(0.5cm,0)$);
  \draw[->] (DS1) -- (DS1a) -- node[above,sloped,pos=0.3] {12:[else]/NACK} (RDY);
  \draw[->] (D1) -- node[right] {2:[R]} (DS1); % DIR set

  \node (DS2) [diamond, draw] at (RC) {};
  \node (DS2b) [right=0 of DS2, diamond, draw] {};
  \draw[->] (DS2) -- node[left,pos=0.3] {15:[STOP]} (RDY);


  % wait for data sent
  \node (WDS) [draw, rounded corners=15pt,
               minimum height=1.0cm, minimum width=2.0cm, thick]
        at ($0.6*(RL)+0.4*(RC)$) 
        {\parbox[c]{0.9in}{\centering{WAIT for DATA SENT}}};

  \draw[->,rounded corners]
        (DS1) -- (RL) -- node[above,pos=-0.4] {11:[OK]/ACK,put} (WDS);

  \draw[->] (WDS) -- node[above] {13:intr} (DS2);

  \node (WFS) [right=2.0cm of RDY,rounded corners=15pt,
    draw, minimum height=1.0cm, minimum width=2.0cm, thick]
        {\parbox[c]{0.8in}{\centering{WAIT for STOP{ }$|$Sr}}};

  \coordinate (WB1) at ($(WFS)+(-0.8cm,-1.0cm)$);
  \coordinate (WB2) at ($(RDY.west)+(-0.8cm,-1.0cm)$);
  \coordinate (WB3) at ($(J1x)+(0.5cm,0)$);
  \draw[->,rounded corners] (WFS) -- (WB1) --
       node [above,pos=0.3] {97:intr[ADR]} (WB2) -- (WB3) -- (J1x);

  \node (FLD) [right=2.0cm of WFS,rounded corners=15pt,
    draw, minimum height=0.4in, minimum width=0.75in, thick]
        {\parbox[c]{0.8in}{\centering{FAILED}}};

  \node (DS3) at (DS2 -| WFS) [diamond,draw] {};

  \draw[->] (DS2b) -- node[above] {14:[DIF]} (DS3);

  \draw[->] (DS3) -- node[right,pos=0.1]{18:[NACK]} (WFS);
  \draw[->] (WFS) -- node[below,sloped] {4:intr[STOP]} (RDY);

  \coordinate (SE1) at ($(RDY -| DS2b)+(0.0,+1.5cm)$);
  \coordinate (SE3) at ($(FLD)+(0.0,+1.5cm)$);
  \coordinate (SE2) at ($0.1*(SE1)+0.9*(SE3)$);
  \draw[->,rounded corners] (DS2b) -- node[right] {29:[err]} (SE1)
     -- (SE2) -- (FLD);

  \node (RD4) [above=0.7cm of DS3, diamond, draw] {};
  \draw[->] (DS3) -- node[right] {17:[ACK]/getnext} (RD4);

  \coordinate (R3) at ($(DS3)+(0.0cm,-0.4cm)$);
  
  %\draw[->,rounded corners]
  %(DS3a) -- (R3) -- node[above,sloped] {19:[STOP]} (RDY);

  \coordinate (RX1) at ($(RD4) + (0,0.7cm)$);
  \draw[->,rounded corners] (RD4) -- (RX1)
            -- node[below] {20:[valid]/put} (RX1 -| WDS) -- (WDS);

  % put zero out
  \draw[-] (RD4) -- node[below] {21:[else]/zero} (RD4 -| WDS);

  % --- WRITE -----------------------

  \coordinate (WC) at ($(RDY) + (0,-2.5cm)$);
  \coordinate (WL) at (D1 |- WC);

  \node (DR1) [below=0.9cm of D1, diamond, draw] {};
  \draw[->] (D1) -- node[right] {3:[W]} (DR1); % DIR clr
  \coordinate (DR1a) at ($(DR1.east)+(0.5cm,0)$);
  \draw[->,rounded corners]
   (DR1) -- (DR1a) -- node[below,sloped,pos=0.2] {32:[else]/NACK} (RDY);

  \node (DR2) [diamond, draw] at (WC) {};
  \draw[->] (DR2) -- node[left,pos=0.2] {34:[STOP]} (RDY);

  % wait for data
  \node (WDR) [draw, rounded corners=15pt,
               minimum height=0.4in, minimum width=0.75in, thick]
        at ($0.6*(WL)+0.4*(WC)$) 
        {\parbox[c]{0.9in}{\centering{WAIT for DATA RCVD}}};

  \draw[->,rounded corners]
        (DR1) -- (WL) -- node[below,pos=-0.4] {31:[OK]/ACK} (WDR);

  \draw[->] (WDR) -- node[below] {33:intr} (DR2); % can be APIF or DIF

  \node (DR2b) [right=0 of DR2, diamond, draw] {};

  \node (DR3) at (DR2 -| WFS) [diamond,draw] {};

  \coordinate (W1) at ($(DR2b) + (0,-1.0cm)$);
  %\draw[->,rounded corners]
  %(DR2b) -- node[right] {35:[OK]/get,ACK} (W1) -- (WDR |- W1) -- (WDR);

  \draw[->] (DR3) -- node[right,pos=0.1] {36:[else]/NACK} (WFS);

  \draw[->] (DR2b) -- node[above] {99:[DIF]} (DR3);

  \coordinate (RE1) at ($(RDY -| DR2b)+(0.0,-1.5cm)$);
  \coordinate (RE3) at ($(FLD)+(0.0,-1.5cm)$);
  \coordinate (RE2) at ($0.1*(RE1)+0.9*(RE3)$);
  \draw[->,rounded corners] (DR2b) -- node[right] {98:[err]} (RE1)
     -- (RE2) -- (FLD);

  % failure return
  \coordinate (FR1) at ($(RDY -| DS2b)+(0.5cm,+1.1cm)$);
  \coordinate (FR3) at ($(FLD)+(0.0,+1.1cm)$);
  \coordinate (FR2) at ($0.3*(FR1)+0.7*(FR3)$);
  \draw[->,rounded corners] (FLD) -- (FR2)
       -- node[below,pos=0.1] {reset} (FR1) -- (RDY);
  
  \node (sig) [below=1cm of WL,rectangle,draw] {\tiny MattRW v230227a};
\end{tikzpicture}

\iffalse
Notes:
\begin{itemize}
BUSERR logic only works if the TWI Master is enabled.
\end{itemize}

\fi
