% pins.tex
%
% $ pdflatex pins.tex
% $ convert -density 300 pins.pdf -quality 90 pins.png

%%\documentclass[crop,tikz,convert=pdf2svg]
%\documentclass[preview]{standalone}
\documentclass{article}
\usepackage{tikz}

%\usetikzlibrary{shapes,angles,arrows,calc,positioning,matrix}

\begin{document}

Currently the only kind of pin we deal with is iopin.
An iopin can provide digital or analog behavior.

\leftline{iopin use cases:}
\begin{enumerated}
\item input, output (or output high-impedance?)
\item input levels: hi, lo
\item input events: rising, falling, either
\item read output (i.e., can always read)
\item analog input or output
\end{enumerated}

\leftline{controls:}
\begin{enumerated}
\item slew rate control
\item pullup for output
\item disable clock sync (to main osc) to reduce power
\item sync or async input event generation
\item port invdr (input or output)
\end{enumerated}

\leftline{states:}
\begin{enumerated}
\item aveage power consumption
\end{enumerated}

\leftlne{port registers:}
\begin{enumerated}
\item DIR, DIRSET, DIRCLR, DIRTGL
\item OUT, OUTSET, OUTCLR, OUTTGL
\item IN
\item INTFLAGS
\item PORTCTL
  one bit to enable slew-rate limiting
\item PINnCTRL
  invert, pullup enable, interrupt on edge(s) or level, or none, and
  input disable
\end{enumerated}

\leftline{model:}
\begin{align*}
  Ip & = (Vdd - Vn)/Rp \\
  %if (OUT 
  Iin & = TBD \\
\end{align*}
where $R_p$ is typically 35 kOhm (range 20 - 50).
In analog mode $R_{in}$ is 14 kOhm.

\vskip 5pt

To bus, pins are pulling low, high imped, or pullup.
If any pulling low, then all Vp are 0.0.
else, if any pullup, then all Vp are Vdd.

\end{document}
% --- last line ---
